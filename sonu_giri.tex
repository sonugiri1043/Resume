%-------------------------
% Resume in Latex
% Author : Sonu Giri
% License : MIT
%------------------------

\documentclass[letterpaper,11pt]{article}

\usepackage{latexsym}
\usepackage[empty]{fullpage}
\usepackage{titlesec}
\usepackage{marvosym}
\usepackage[usenames,dvipsnames]{color}
\usepackage{verbatim}
\usepackage{enumitem}
\usepackage[hidelinks]{hyperref}
\usepackage{fancyhdr}
\usepackage[english]{babel}
\usepackage{tabularx}
\input{glyphtounicode}

\pagestyle{fancy}
\fancyhf{} % clear all header and footer fields
\fancyfoot{}
\renewcommand{\headrulewidth}{0pt}
\renewcommand{\footrulewidth}{0pt}

% Adjust margins
\addtolength{\oddsidemargin}{-0.5in}
\addtolength{\evensidemargin}{-0.5in}
\addtolength{\textwidth}{1in}
\addtolength{\topmargin}{-.5in}
\addtolength{\textheight}{1.0in}

\urlstyle{same}

\raggedbottom
\raggedright
\setlength{\tabcolsep}{0in}

% Sections formatting
\titleformat{\section}{
  \vspace{-4pt}\scshape\raggedright\large
}{}{0em}{}[\color{black}\titlerule \vspace{-5pt}]

% Ensure that generate pdf is machine readable/ATS parsable
\pdfgentounicode=1

%-------------------------
% Custom commands
\newcommand{\resumeItem}[2]{
  \item\small{
    \textbf{#1}{: #2 \vspace{-2pt}}
  }
}

% Just in case someone needs a heading that does not need to be in a list
\newcommand{\resumeHeading}[4]{
    \begin{tabular*}{0.99\textwidth}[t]{l@{\extracolsep{\fill}}r}
      \textbf{#1} & #2 \\
      \textit{\small#3} & \textit{\small #4} \\
    \end{tabular*}\vspace{-5pt}
}

\newcommand{\resumeSubheading}[4]{
  \vspace{-1pt}\item
    \begin{tabular*}{0.97\textwidth}[t]{l@{\extracolsep{\fill}}r}
      \textbf{#1} & #2 \\
      \textit{\small#3} & \textit{\small #4} \\
    \end{tabular*}\vspace{-5pt}
}

\newcommand{\resumeSubSubheading}[2]{
    \begin{tabular*}{0.97\textwidth}{l@{\extracolsep{\fill}}r}
      \textit{\small#1} & \textit{\small #2} \\
    \end{tabular*}\vspace{-5pt}
}

\newcommand{\resumeSubItem}[2]{\resumeItem{#1}{#2}\vspace{-4pt}}

\renewcommand{\labelitemii}{$\circ$}

\newcommand{\resumeSubHeadingListStart}{\begin{itemize}[leftmargin=*]}
\newcommand{\resumeSubHeadingListEnd}{\end{itemize}}
\newcommand{\resumeItemListStart}{\begin{itemize}}
\newcommand{\resumeItemListEnd}{\end{itemize}\vspace{-5pt}}

%-------------------------------------------
%%%%%%  CV STARTS HERE  %%%%%%%%%%%%%%%%%%%%%%%%%%%%


\begin{document}

%----------HEADING-----------------
\begin{tabular*}{\textwidth}{l@{\extracolsep{\fill}}r}
  \textbf{\href{https://www.linkedin.com/in/sonugiri/}{\Large Sonu Giri}} & Email : \href{mailto:sonugiri1043@gmail.com}{sonugiri1043@gmail.com}\\
  \href{https://www.linkedin.com/in/sonugiri/}{https://www.linkedin.com/in/sonugiri/} & Mobile : +91 888-490-6153 \\
\end{tabular*}


%-----------EDUCATION-----------------
\section{Education}
  \resumeSubHeadingListStart
    \resumeSubheading
      {Indian Institute of Technology (IIT)}{Ropar, India}
      {Bachelor of Technology in Computer Science and Engineering;  CGPA: 9.37/10.0}{Aug. 2009 -- May. 2013}
    \resumeSubheading
      {Army School}{Kota, India}
      {Senior Secondary; 94.40/100}{2008}
  \resumeSubHeadingListEnd


%-----------EXPERIENCE-----------------
\section{Experience}
  \resumeSubHeadingListStart

    \resumeSubheading
      {PayPal}{Bangalore, India}
      {MTS, T25}{Sept 2020 - Present}
      \resumeItemListStart
        \resumeItem{Network Health and Endpoint Monitoring Solution (Camelot)}
          {I lead the design, code and deployment of Camelot. Camelot is designed to provide Experience Centric Network Monitoring by mimicking application behavior in agents. It’s capable of monitoring network connectivity and latency matrix for PayPal devices as well as external endpoints. It can automatically detect if there is a connectivity or latency issue in network and notify Network Operations team. Tech used GoLang, Redis, Kubernetes, Docker, InfluxDb, Grafana, LoadBalancer, mTLS.}
        \resumeItem{Continuous Development \& Continuous Deployment and Auto CodeGen Infrastructure}
          {This project helped the team migrate from monolithic application framework to true Micro Services architecture with API Gateway and auto generated code. With this the deployment cycles have reduced from days to minutes and the whole team is benefiting from it. This involved Dockizing all CNS services, setting up Argo Continuous deployment pipeline and sealed secret in Kubernetes cluster. Setting up infra like Nginx, API gateway, load balancer and services on Kubernetes cluster. The scope of work also involved automating Jenkins’s pipeline to automate deployment and code generation for UI. Tech used: Kubernetes, Argo, Sealed Secret, Jinja Template, Javascript, Jenkins, MySQL. }
        \resumeItem{Edge Traffic Engineering Deployment and UI}
          {The goal of this project was to build a RESTful service and UI to manage edge components. Earlier Edge team was using python based CLI to manage edge components like CDN, PoP, Borders. I developed UI for the edge traffic management service and contributed to packaging and deployment of the software. Helped onboard this software to Kubernetes cluster. Tech used: Javascript}
        \resumeItem{Firewall Ticket Automation }
          {Developed a tool with the aim of providing an easy-to-use CLI tool to the engineers. Network Security engineers were able to evaluate tickets in few minutes which was taking up to 30mins earlier. In addition to this usage, I also developed accountability among the tooling by providing dashboard to get weekly/monthly report on firewall tickets processed thru automation. This provided visibility to the leadership. Tech used: Python, Grafana, MySQL}
      \resumeItemListEnd
      
% --------Multiple Positions Heading------------
%    \resumeSubSubheading
%     {Software Engineer I}{Oct 2014 - Sep 2016}
%     \resumeItemListStart
%        \resumeItem{Apache Beam}
%          {Apache Beam is a unified model for defining both batch and streaming data-parallel processing pipelines}
%     \resumeItemListEnd
%    \resumeSubHeadingListEnd
%-------------------------------------------

    \resumeSubheading
      {Arista Networks}{Bangalore, India}
      {Software Engineer}{July 2013 - Sept 2020}
      \resumeItemListStart
        \resumeItem{Intro}
          {Worked on Arista’s EOS, which is a Linux based Operating System that runs on Arista switches. Some relevant projects taken at Arista are listed below:}
        \resumeItem{Terraform Provider}
          {Terraform relies on plugins called "providers" to interact with cloud providers. Developed terraform provider and terraform scripts for Arista vEOS. This enables automatic provisioning of Arista virtual router deployment in public cloud (AWS and Azure).}
        \resumeItem{Traffic Classification Library}
          {Created traffic classifier library for traffic classification based on 5 tuple(src/dst IP, src/dst port and protocol). This is used by applications like Dynamic Path steering.}
        \resumeItem{File Replication Library}
          {Developed file replication library that replicates file[s] from source to destination whenever there is a change to a file. Used ’inotify’ and ’rsync’ utilities internally. It offers various modes of replication like periodic, inotify mode. This was used for license file replication across HA Cluster.}
        \resumeItem{Zone Based Firewall}
          {Developed an Iptables based Firewall on linux platform. A zone groups together interfaces/subnets and applies policy across zones.}
        \resumeItem{Traffic Mirroring}
          {Developed Traffic Mirroring solution for a linux based router. Used Iptables and linux ’tc’ utility provided by linux platform.}
        \resumeItem{Flow Sampler on Linux}
          {Used Iptables to create a flow sampler on a linux based router.}
        \resumeItem{IP Path Tracer for O.A.M.}
          {Implemented an IP Path Tracer for Operation and Management (OAM). OAM functions are important for fault management and performance monitoring. Based on OpenFlow and VXLAN.}
        \resumeItem{Audio Video Bridging Test Infra}
          {Used Intel open sourced open-AVB code to develop test infra for Arista AVB feature. At high level, this involved modifying Intel’s code to suit Arista’s needs, creating different network namespaces to simulate test device and providing network connectivity to them.}
      \resumeItemListEnd

    \resumeSubheading
      {Aston University}{birmingham , UK}
      {Research Intern and B.Tech Project}{May 2012 - April 2013}
      \resumeItemListStart
        \resumeItem{Artificial Landscape Generator}
          {Constructed open-source tool to generate 2D Natural and Urban Landscapes for testing ecological theories. It involved using machine learning techniques to learn from real data and use it to generate artificial landscape.}
      \resumeItemListEnd

  \resumeSubHeadingListEnd


%
%--------PROGRAMMING SKILLS------------
\section{Programming Skills}
  \resumeSubHeadingListStart
    \resumeSubItem{Languages}
    {C++, C, Python, GoLang, SQL}
    \resumeSubItem{Technologies}
      {Kubernetes, Docker, Terraform, Jenkins, AWS, GCP, Azure, Redis, InfluxDb, Grafana}
  \resumeSubHeadingListEnd

%-------------------------------------------
\end{document}

